\section{Συμπεράσματα}
Αυτή η ενότητα παρουσιάζει τα αποτελέσματα απόδοσης του αλγορίθμου MinBFT-SGX με τη χρήση micro-benchmarks. Μετρήσαμε την καθυστέρηση και την απόδοση των εφαρμογών MinBFT με μηδενικές λειτουργίες. Το PBFT θεωρείται συχνά ως η γραμμή βάσης για τους αλγόριθμους BFT, επομένως μας ενδιαφέρει η σύγκριση της δικής μας υλοποίησης του αλγορίθμου με την εφαρμογή που είναι διαθέσιμη στο web. Για να συγκρίνουμε αυτή την εφαρμογή με τον αλγόριθμο MinBFT-SGX, χρησιμοποίησα την υλοποίηση της κανονικής λειτουργίας του PBFT στην Java (JPBFT).
\begin{itemize}
\item \textbf{Αύξηση Απόδοσης}\\
Η Intel από το 2015 που ανακοίνωσε την τεχνολογία Software Guard Extensions, από την έκτη γενιά επεξεργαστών που βασίζεται στην μικροαρχιτεκτονική Intel Skylake και μετέπειτα την ενσωματώνει δωρεάν σε κάθε νέα γενιά επεξεργαστών. Αυτό σημαίνει πως η τεχνολογία αυτή βρίσκεται σε κάθε προσωπικό υπολογιστή και τα οφέλη της υλοποίησης μας είναι προσβάσιμα από όλους.
\item \textbf{Διαθεσιμότητα Intel SGX}, Η Intel από το 2015 που ανακοίνωσε την τεχνολογία Software Guard Extensions, από την έκτη γενιά επεξεργαστών που βασίζεται στην μικροαρχιτεκτονική Intel Skylake και μετέπειτα την ενσωματώνει δωρεάν σε κάθε νέα γενιά επεξεργαστών. Αυτό σημαίνει πως η τεχνολογία αυτή βρίσκεται σε κάθε προσωπικό υπολογιστή και τα οφέλη της υλοποίησης μας είναι προσβάσιμα από όλους.
\end{itemize}


\section{Μελλοντική Δουλειά}
Αυτή η ενότητα παρουσιάζει τα αποτελέσματα απόδοσης του αλγορίθμου MinBFT-SGX με τη χρήση micro-benchmarks. Μετρήσαμε την καθυστέρηση και την απόδοση των εφαρμογών MinBFT με μηδενικές λειτουργίες. Το PBFT θεωρείται συχνά ως η γραμμή βάσης για τους αλγόριθμους BFT, επομένως μας ενδιαφέρει η σύγκριση της δικής μας υλοποίησης του αλγορίθμου με την εφαρμογή που είναι διαθέσιμη στο web. Για να συγκρίνουμε αυτή την εφαρμογή με τον αλγόριθμο MinBFT-SGX, χρησιμοποίησα την υλοποίηση της κανονικής λειτουργίας του PBFT στην Java (JPBFT).