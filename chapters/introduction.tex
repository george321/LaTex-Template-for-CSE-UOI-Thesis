Υπενθυμίζουμε ότι ένα blockchain απαιτεί ένα μηχανισμό για την επίτευξη κατανεμημένης συμφωνίας ή την επικύρωση και την κανονικοποίηση ενός μόνο καταλόγου. Το πρωτόκολλο BFT αναφέρεται στο χαρακτηριστικό των κατανεμημένων συστημάτων που τους επιτρέπει να φτάσουν σε συμφωνία ενάντια στα βυζαντινά σφάλματα, δηλαδή σε καταστάσεις όπου τα συστατικά του συστήματος θα αποτύχουν - αλλά όχι μόνο να αποτύχουν - οι εσφαλμένοι βυζαντινοί κόμβοι θα ενεργήσουν αυθαίρετα και συχνά παρουσιάζουν αντικρουόμενες πληροφορίες σε διαφορετικούς κόμβους του συστήματος.

Υπενθυμίζουμε ότι ένα blockchain απαιτεί ένα μηχανισμό για την επίτευξη κατανεμημένης συμφωνίας ή την επικύρωση και την κανονικοποίηση ενός μόνο καταλόγου. Το πρωτόκολλο BFT αναφέρεται στο χαρακτηριστικό των κατανεμημένων συστημάτων που τους επιτρέπει να φτάσουν σε συμφωνία ενάντια στα βυζαντινά σφάλματα, δηλαδή σε καταστάσεις όπου τα συστατικά του συστήματος θα αποτύχουν - αλλά όχι μόνο να αποτύχουν - οι εσφαλμένοι βυζαντινοί κόμβοι θα ενεργήσουν αυθαίρετα και συχνά παρουσιάζουν αντικρουόμενες πληροφορίες σε διαφορετικούς κόμβους του συστήματος.

Η εργασία αυτή εστίασε στα BFT πρωτόκολλα που όπως αναφέρθηκε παραπάνω τα περισσότερα permissioned blockchains χρησιμοποιούν για μηχανισμό συμφωνίας. Για παράδειγμα, το Hyperledger της εταιρείας IBM \cite{hyperledgeribm} χρησιμοποιεί το Practical Byzantine Fault Tolerance (PBFT) \cite{pbft} για μηχανισμό συμφωνίας. Παρόλο που το PBFT μπορεί να επιτύχει μεγαλύτερη απόδοση από το μηχανισμό του Bitcoin ακόμη δεν μπορεί να φτάσει τις αποδόσεις των υπαρχόντων μεθόδων συναλλαγής (π.χ η Visa διαχειρίζεται δεκάδες χιλιάδες συναλλαγές το δευτερόλεπτο). Επιπλέον το PBFT μπορεί να επεκταθεί μόνο σε μερικές δεκάδες κόμβων αφού ανταλλάσει $O(n^{2})$ μηνύματα για την επίτευξη συμφωνίας για μία λειτουργία μεταξύ $n$ διακομιστών. Έτσι, η βελτίωση της επεκτασιμότητας και της απόδοσης των πρωτοκόλλων BFT είναι απαραίτητη για την εξασφάλιση της καθιέρωσης τους σε υπάρχουσες λύσεις blockchain.

Σε αυτή την εργασία, βασιστήκαμε σε μία υπάρχουσα υλοποίηση του BFT, στον αλγόριθμο MinBFT \cite{minbftpaper}. Είναι μεταγενέστερο του PBFT και βελτιώνει στο ότι χρειάζεται λιγότερους διακομιστές καθώς και ότι ανταλλάσει λιγότερα μηνύματα. Το MinBFT είναι ένας αλγόριθμος βυζαντινής συμφωνίας ο οποίος απαιτεί $2f+1$ διακομιστές για $f$ ελαττωματικούς (Βυζαντινούς) διακομιστές. Οι BFT αλγόριθμοι τυπικά χρειάζονται $3f+1$ διακομιστές όμως με την βοήθεια ενός μηχανισμού ασφαλείας που υπάρχει σε κάθε υπολογιστή που εκτελεί το MinBFT ο αριθμός των διακομιστών μπορεί να πέσει απο $3f+1$ σε $2f+1$. Αυτός ο μηχανισμός ασφαλείας πρέπει να λειτουργεί σωστά ακόμα και αν όλο το σύστημα έχει παραβιαστεί. Το βασικό πρόβλημα είναι ότι όλοι οι διακομιστές θα πρέπει να εκτελέσουν την ίδια ακολουθία λειτουργιών. Ο μηχανισμός ασφαλείας θα πρέπει να παρέχει μια υπηρεσία η οποία θα ορίζει την ακολουθία των λειτουργιών που εκτελούν,  με τέτοιο τρόπο όπου ένας κακόβουλος διακομιστής δεν θα μπορεί να κάνει τους άλλους σωστούς διακομιστές να εκτελέσουν μια άλλη λειτουργία αντί για την κ-οστη λειτουργία. Αυτό μπορεί να υλοποιηθεί με την βοήθεια ενός έμπιστου μονοτονικού μετρητή που θα αντιστοιχίζει ακολουθιακούς αριθμούς σε κάθε λειτουργία. Στην αρχική έκδοση του MinBFT \cite{minbftpaper} είχε χρησιμοποιηθεί ο μηχανισμός ασφαλείας Atmel TPM 1.2 για την υλοποίηση της υπηρεσίας USIG (βλ. ενότητα \ref{usigSection}) η οποία είναι υπεύθυνη για την δημιουργία του μονοτονικού μετρητή. 


\section{Στόχοι}
Υπενθυμίζουμε ότι ένα blockchain απαιτεί ένα μηχανισμό για την επίτευξη κατανεμημένης συμφωνίας ή την επικύρωση και την κανονικοποίηση ενός μόνο καταλόγου. Το πρωτόκολλο BFT αναφέρεται στο χαρακτηριστικό των κατανεμημένων συστημάτων που τους επιτρέπει να φτάσουν σε συμφωνία ενάντια στα βυζαντινά σφάλματα, δηλαδή σε καταστάσεις όπου τα συστατικά του συστήματος θα αποτύχουν - αλλά όχι μόνο να αποτύχουν - οι εσφαλμένοι βυζαντινοί κόμβοι θα ενεργήσουν αυθαίρετα και συχνά παρουσιάζουν αντικρουόμενες πληροφορίες σε διαφορετικούς κόμβους του συστήματος.

\section{Δομή της Διπλωματικής Εργασίας} \label{domi}
Υπενθυμίζουμε ότι ένα blockchain απαιτεί ένα μηχανισμό για την επίτευξη κατανεμημένης συμφωνίας ή την επικύρωση και την κανονικοποίηση ενός μόνο καταλόγου. Το πρωτόκολλο BFT αναφέρεται στο χαρακτηριστικό των κατανεμημένων συστημάτων που τους επιτρέπει να φτάσουν σε συμφωνία ενάντια στα βυζαντινά σφάλματα, δηλαδή σε καταστάσεις όπου τα συστατικά του συστήματος θα αποτύχουν - αλλά όχι μόνο να αποτύχουν - οι εσφαλμένοι βυζαντινοί κόμβοι θα ενεργήσουν αυθαίρετα και συχνά παρουσιάζουν αντικρουόμενες πληροφορίες σε διαφορετικούς κόμβους του συστήματος.